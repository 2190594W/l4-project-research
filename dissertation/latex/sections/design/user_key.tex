\section{User Key Design}
\label{sec:design_user_key}

User keys are the cornerstone of the \theResServer system, functioning as the private decryption keys for all users of the system and identifying an individual through assigned attributes. \Cref{subsec:analysis_abe} describes the role of user keys in the greater \acrshort{abe} system, sections \ref{subsec:analysis_deployment_mks} \& \ref{subsec:analysis_deployment_iuk} describe the process of issuing new user keys and \Cref{sec:analysis_enrolment} describes the enrolment process for a new user receiving their private key.

In designing the \theResServer system, the user keys are issued as relatively small files that a user must then store securely after enrolling into the system. If a user's key becomes compromised then all the resources that key provided access to are also compromised.
\vskip 0.5em
The system provides passive revocation (as described by \citet{Akinyele2011}) through timestamped attributes such as those specified in \Cref{fig:case_study_policy_5} which protect against access being granted to new resources if the user key was created in previous academic years. Full revocation of user keys could also be possible, with the employment of a new online mediator system which would be involved in the decryption process and would block decryption for users that have had their access revoked. \citet{Lewko2008, Narayan2010} show that further revocation methods could also be implemented into the system instead.

Additionally, the design of the system leaves abstract the assignment of a user's attributes, as this was deemed outside of the scope of the project, instead the design relies on members of the \acrshort{dcs} Administration staff to identify and verify a user's attributes. The design does suggest use of the MyCampus/Student Centre services as well as the HR/Payroll system for \acrshort{dcs} students and staff, and further suggests that the system could integrate with the university's Single-Sign-On (SSO) system for quick access to a user's details (SSO and its risks are described by \citet{Pashalidis2003, Tsyrklevich2007}).
