\section{User Key Design}
\label{sec:design_user_key}

User keys are the cornerstone of the \theResServer system, functioning as the private decryption keys for all users of the system and identifying an individual through assigned attributes. \Cref{subsec:analysis_abe} describes the role of user keys in the greater \acrshort{abe} system, and Sections \ref{subsec:analysis_deployment_mks} \& \ref{subsec:analysis_deployment_iuk} describe the process of issuing new user keys.

In designing the \theResServer system, the user keys are issued as small files that a user must then store securely. If a user's key becomes compromised then all the resources that key provided access to, are also compromised.
\vskip 0.5em
The \theResServer system provides passive revocation (as described by \citet{Akinyele2011}) through timestamped attributes such as those specified in \Cref{fig:case_study_policy_5}. Full revocation of user keys could also be possible, with the integration of an online mediator system which would be involved in the decryption process and would block decryption for users that have had their access revoked \citep{Green2011}. Alternatively, \citet{Lewko2008, Narayan2010} show that further revocation methods could be implemented into the system instead.

Additionally, the design of the system leaves abstract the assignment of a user's attributes, as this was deemed outside of the scope of the project. Instead the design relies on members of the \acrshort{dcs} Administration staff to identify and verify a user's attributes. The design does suggest use of the MyCampus/Student Centre services as well as the HR/Payroll system for \acrshort{dcs} students and staff.
