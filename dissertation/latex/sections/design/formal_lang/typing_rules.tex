\subsection{Typing Rules}
\label{subsec:typing-rules}

\begin{figure}[ht]
  \centering
\begin{mathpar}
  \infer*[left=Intro-Int]
  {
    \\
  }
  {
    \ty{i}{\TyInt}
  }
  \and
  \infer*[left=Intro-Bool]
  {
    \\
  }
  {
    \ty{b}{\mathbb{B}}
  }
  \and
  \infer*[left=Intro-Date]
  {
    \\
  }
  {
    \ty{d}{\mathbb{D}}
  }
  \and
  \infer*[left=Intro-String]
  {
    \\
  }
  {
    \ty{s}{\mathbb{S}}
  }
  \and
  \infer*[left=Intro-List]
  {
    \\
  }
  {
    \ty{l}{\mathbb{L}_T}
  }
  \and
  \infer*[left=OR]
  {
    \env{\ty{a}{\mathbb{B}}}\\
    \env{\ty{b}{\mathbb{B}}}
  }
  {
    \env{\ty{\exprOr{a}{b}}{\mathbb{B}}}
  }
  \and
  \infer*[left=AND]
  {
    \env{\ty{a}{\mathbb{B}}}\\
    \env{\ty{b}{\mathbb{B}}}
  }
  {
    \env{\ty{\exprAnd{a}{b}}{\mathbb{B}}}
  }
  \and
  \infer*[left=GT]
  {
    \env{\ty{a}{T}}\\
    \env{\ty{b}{T}}\\
    \left[ T \in \{ \mathbb{Z}, \mathbb{D} \} \right]
  }
  {
    \env{\ty{\exprGT{a}{b}}{\mathbb{B}}}
  }
  \and
  \infer*[left=LT]
  {
    \env{\ty{a}{T}}\\
    \env{\ty{b}{T}}\\
    \left[ T \in \{ \mathbb{Z}, \mathbb{D} \} \right]
  }
  {
    \env{\ty{\exprLT{a}{b}}{\mathbb{B}}}
  }
  \and
  \infer*[left=EQ]
  {
    \env{\ty{a}{T}}\\
    \env{\ty{b}{T}}\\
    \left[ T \in \{ \mathbb{Z}, \mathbb{B}, \mathbb{D}, \mathbb{S}, \mathbb{L}_T \} \right]
  }
  {
    \env{\ty{\exprEQ{a}{b}}{\mathbb{B}}}
  }
  \and
  \infer*[left=GTE]
  {
    \env{\ty{a}{T}}\\
    \env{\ty{b}{T}}\\
    \left[ T \in \{ \mathbb{Z}, \mathbb{D} \} \right]
  }
  {
    \env{\ty{\exprGTE{a}{b}}{\mathbb{B}}}
  }
  \and
  \infer*[left=LTE]
  {
    \env{\ty{a}{T}}\\
    \env{\ty{b}{T}}\\
    \left[ T \in \{ \mathbb{Z}, \mathbb{D} \} \right]
  }
  {
    \env{\ty{\exprLTE{a}{b}}{\mathbb{B}}}
  }
\end{mathpar}
  \caption{\label{fig:rules}The formal definition of the Typing Rules for \thePolicyLang}
\end{figure}

\Cref{fig:rules} presents the \thePolicyLang typing rules.
These rules dictate what it means for an expression/statement to be well-formed within \thePolicyLang and for any expression/statement in \thePolicyLang, we use the typing rules to construct a derivation that provides proof that the expression/statement is well-typed, that is we can apply each rule and form a derivation tree. If we cannot construct this tree then the expression is ill-typed and syntactically not valid.\\
The Typing Rules define 5 base cases for the 5 value types \thePolicyLang supports (as defined in \cref{fig:syntax}), meaning instances of the 5 types derive directly to their type. Next, the boolean logical operators OR \& AND are defined as requiring two Boolean parameters that also derive to a Boolean type return. The standard comparison operators \textbf{greaterThan}, \textbf{lessThan}, \textbf{greaterThanEqual} \& \textbf{lessThanEqual} all take two parameters of a matching type, where the type may be Integer or Date, and derives to a Boolean response. Lastly, the \textbf{equal} comparison similarly takes two paramters of a matching type, but where the type may be Integer, Boolean, Date, String or List, and also derives to a Boolean return.
