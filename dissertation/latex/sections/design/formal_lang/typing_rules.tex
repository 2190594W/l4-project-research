\subsection{Typing Rules}
\label{subsec:typing-rules}

\begin{figure}[ht]
  \centering
\begin{mathpar}
  \infer*[left=Intro-Nat]
  {
    \\
  }
  {
    \ty{n}{\TyInt}
  }
  \and
  \infer*[left=Intro-Boolean]
  {
    \\
  }
  {
    \ty{b}{\mathbb{B}}
  }
  \and
  \infer*[left=Intro-Date]
  {
    \\
  }
  {
    \ty{d}{\mathbb{D}}
  }
  \and
  \infer*[left=Intro-String]
  {
    \\
  }
  {
    \ty{s}{\mathbb{S}}
  }
  \and
  \infer*[left=Intro-List]
  {
    \\
  }
  {
    \ty{l}{\mathbb{L}_T}
  }
  \and
  \infer*[left=OR]
  {
    \env{\ty{a}{\mathbb{B}}}\\
    \env{\ty{b}{\mathbb{B}}}
  }
  {
    \env{\ty{\exprOr{a}{b}}{\mathbb{B}}}
  }
  \and
  \infer*[left=AND]
  {
    \env{\ty{a}{\mathbb{B}}}\\
    \env{\ty{b}{\mathbb{B}}}
  }
  {
    \env{\ty{\exprAnd{a}{b}}{\mathbb{B}}}
  }
  \and
  \infer*[left=GT]
  {
    \env{\ty{a}{T}}\\
    \env{\ty{b}{T}}\\
    \left[ T \in \{ \mathbb{N}, \mathbb{D} \} \right]
  }
  {
    \env{\ty{\exprGT{a}{b}}{\mathbb{B}}}
  }
  \and
  \infer*[left=LT]
  {
    \env{\ty{a}{T}}\\
    \env{\ty{b}{T}}\\
    \left[ T \in \{ \mathbb{N}, \mathbb{D} \} \right]
  }
  {
    \env{\ty{\exprLT{a}{b}}{\mathbb{B}}}
  }
  \and
  \infer*[left=EQ]
  {
    \env{\ty{a}{T}}\\
    \env{\ty{b}{T}}\\
    \left[ T \in \{ \mathbb{N}, \mathbb{B}, \mathbb{D}, \mathbb{S} \} \right]
  }
  {
    \env{\ty{\exprEQ{a}{b}}{\mathbb{B}}}
  }
  \and
  \infer*[left=GTE]
  {
    \env{\ty{a}{T}}\\
    \env{\ty{b}{T}}\\
    \left[ T \in \{ \mathbb{N}, \mathbb{D} \} \right]
  }
  {
    \env{\ty{\exprGTE{a}{b}}{\mathbb{B}}}
  }
  \and
  \infer*[left=LTE]
  {
    \env{\ty{a}{T}}\\
    \env{\ty{b}{T}}\\
    \left[ T \in \{ \mathbb{N}, \mathbb{D} \} \right]
  }
  {
    \env{\ty{\exprLTE{a}{b}}{\mathbb{B}}}
  }
\end{mathpar}
  \caption{\label{fig:rules}Typing Rules}
\end{figure}

\Cref{fig:rules} present's the \thePolicyLang typing rules.
These rules dictate what it means for an expression/statement to be well-formed.
We do this by assigning types to expressions/statements.
We read typing rules as follows: Things above the lines are premises such that if all premises are true then the judgement (below the line) will also be true.
When given any expression/statement in our language we can use the typing rules to construct a derivation that provides proof that the expression/statement is well-typed, that is we can apply each rule and form a derivation tree.
If we cannot construct this tree then the expression is ill-typed and syntactically not valid.
Typing rules are a compile time static check.

We can only proceed to computation/evaluation of our language if it is well-typed.
