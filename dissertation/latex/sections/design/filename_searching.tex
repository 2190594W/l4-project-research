\section{Filename Searching}
\label{sec:design_file_search}

Users of the \theResServer system need to be able to identify resources they wish to download from the \acrfull{prs}. This may be trivial at first, since when there are only a few resources uploaded a user can easily browse a list to find the resources they need, however this quickly becomes infeasible as the \acrshort{prs} starts to store \textit{100s} of resources. Given the scale of the \acrfull{dcs}, this will become an issue quickly and so an alternative method of finding resources is required. Hence, we introduce a search utility for the \acrshort{prs}.
\vskip 0.5em
As described in \cref{subsec:design_resources}, the \acrshort{prs} stores the metadata of uploaded resources in a local database to maintain a complete unawareness of the contents of any resources. An advantage to this design, is that the database can be efficiently queried for complex searches over all resources uploaded, requiring no storage operations by the \acrshort{prs}'s OS. Additionally, in future the metadata stored for each resource can easily be expanded to meet the evolving needs of the \acrshort{dcs}.\\
The \acrfull{crs} implements the search utilities offered by the \acrshort{prs}, offering resource searching directly to the user, with the added benefit of obscuring resources that the user would not be able to decrypt. This is possible as the \acrshort{crs} is aware of the user's assigned attributes and the \acrshort{prs} records the policy of every resource in its database at upload-time, allowing for quick checks of a user's attributes against a resource's policy.
\vskip 0.5em
The \theResServer system is designed to implement a simple filename search utility with the possibility to extend searching to include any of the other metadata stored for resources in the future.\\
The filename searching itself is to operate in one of two methods, either by directly matching a slice of a filename and returning a sorted list of the most likely matches (e.g. the query \textit{``report.pdf''} would match the filename \texttt{one\_example\_report.pdf} but not \texttt{one\_example\_report.docx}) or in a fuzzy finder method which would more generally score the similarities between a query and all available filenames, returning a sorted list of the most likely matches (e.g. the query \textit{``pdf apple report four''} would match highly with the filenames \texttt{report\_hci\_four\_final.pdf} \& \texttt{rep\_on\_four\_apples.docx}).
