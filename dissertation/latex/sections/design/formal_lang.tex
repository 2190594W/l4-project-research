\section{Formal Language Definition}
\label{sec:formal_lang}

We offer the following formal language definition for the \theResServer system's Attribute-Based Encryption policy language, \thePolicyLang. The language is designed around the Case Studies described in \cref{sec:analysis_case_studies} and follows an extensible design principle that aims to provide a complete policy solution for the Department of Computing Science.\\
\thePolicyLang directly provides the tools for policies to be constructed for the \cref{sec:analysis_case_studies} Case Studies, which can then be interpreted to the policy format that the \PyOpenABE python bindings (described in \cref{sec:bkgr_openabe}) require.
\vskip 0.5em
\thePolicyLang offers Boolean comparisons for Integer, String \& Date attributes with further support for Lists of multiple values. Additional comparison operations are provided for Integer and Date comparisons with `greater than', `less than' and `equivalent to' all supported, whereas Booleans, Strings and Lists support `equivalent to' operations only.\\
This allows for Integer comparisons such as ``\textit{studentLevel(\textbf{s}) $\geq$ 4}'' (as found in Case Study \#2, \cref{fig:case_study_policy_2}) and Date comparisons such as ``\textit{classRepFrom(\textbf{s}) $\geq$ 17 Sep 2019}'' (as found in Case Study \#5, \cref{fig:case_study_policy_5}). String equivalence operations allow for comparisons such as ``\textit{jobField(\textbf{s}) $\equiv$ `Research \& Teaching'}'' (as found in Case Study \#1, \cref{fig:case_study_policy_1}).\\
List equivalence operations are also supported, allowing for comparisons such as ``\textit{enrolledCourses(\textbf{s}) $\equiv$ [1001, 1017]}'' (as found in Case Study \#2 \cref{fig:case_study_policy_2}) where the policy requires both that `\textit{enrolledCourses(\textbf{s})}' resolves to a list and that that list contains the elements in `\textit{[1001, 1017]}'.

\section{Abstract Syntax, Types, \& Contexts}\label{sec:defs}

\begin{figure}[ht]
  \centering
\begin{align*}
  \ty{n}{\mathbb{Z}}
  &
    \Coloneqq
    \text{Integers}
  & \text{Values}
  \\
  \ty{b}{\mathbb{B}}
  & \Coloneqq
    \mathsf{False}\alt\mathsf{True}
  \\
  \ty{d}{\mathbb{D}}
  & \Coloneqq
    \text{Dates}
  \\
  \ty{s}{\mathbb{S}}
  & \Coloneqq
    \text{Strings}
  \\
  v
  &
    \Coloneqq
    n
    \alt
    b
    \alt
    d
    \alt
    s
  \\
  \ty{l}{\mathbb{L}_T}
  & \Coloneqq
    \emptyset_T\alt{}v\Cons_T{}l
  \\
  e
  &
    \Coloneqq
    v
    \alt
    l
  & \text{Expressions}
  \\
  &
    \firstAlt
    \exprEQ{e}{e}
    \alt
    \exprGT{e}{e}
    \alt
    \exprLT{e}{e}
  \\
  & \firstAlt
    \exprGTE{e}{e}
    \alt
    \exprLTE{e}{e}
  &
  \\
  & \firstAlt
    \exprOr{e}{e}
    \alt
    \exprAnd{e}{e}
  &
  \\
  &
    \firstAlt
    \left( \lambda \mu \bullet e \right)
    \alt
    e \; \$ \; e
  &
    \text{Statements}
  \\
  T
  &
    \Coloneqq
    \mathbb{Z}
    \alt
    \mathbb{B}
    \alt
    \mathbb{D}
    \alt
    \mathbb{S}
    \alt
    \mathbb{L}
    \alt
    {T \rightarrow{} T}
  &
    \text{Types}
  \\
  \Gamma
  &
    \Coloneqq
    \envAdd{(\ty{x}{T})}
    \alt
    \emptyset
    &
      \text{Context}
\end{align*}
  \caption{\label{fig:syntax}The policy language's abstract syntax, types, and context.}
\end{figure}

\Cref{fig:syntax} presents the syntactical structure, and types for our language.
Our language contains Integers, Boolean, Date, String and List values, we leave abstract how integers and dates are written.

Integers \& Dates can be compared using standard comparison operations of: \texttt{greater-than}, \texttt{less-than} and \texttt{equals-to}, as well as \texttt{greater-than-or-equal-to} and \texttt{less-than-or-equal-to}.
Boolean operators provide logical conjunction and disjunction.

A context ($\Gamma$) keeps track of well-typed expressions, and our context can be expanded.


\subsection{Typing Rules}
\label{subsec:typing-rules}

\begin{figure}[ht]
  \centering
\begin{mathpar}
  \infer*[left=Intro-Int]
  {
    \\
  }
  {
    \ty{i}{\TyInt}
  }
  \and
  \infer*[left=Intro-Bool]
  {
    \\
  }
  {
    \ty{b}{\mathbb{B}}
  }
  \and
  \infer*[left=Intro-Date]
  {
    \\
  }
  {
    \ty{d}{\mathbb{D}}
  }
  \and
  \infer*[left=Intro-String]
  {
    \\
  }
  {
    \ty{s}{\mathbb{S}}
  }
  \and
  \infer*[left=Intro-Empty]
  {
    \env{\ty{a}{T}}\\
    \left[ T \in T_l \right]
  }
  {
    \env{\ty{\emptyset_a}{\mathbb{L}_a}}
  }
  \and
  \infer*[left=Intro-Cons]
  {
    \env{\ty{v}{a}}\\
    \env{\ty{l}{\mathbb{L}_a}}\\
    \env{\ty{a}{T}}\\
    \left[ T \in T_l \right]
  }
  {
    \env{\ty{v\Cons_t{}l}{\mathbb{L}_a}}
  }
  \and
  \infer*[left=OR]
  {
    \env{\ty{a}{\mathbb{B}}}\\
    \env{\ty{b}{\mathbb{B}}}
  }
  {
    \env{\ty{\exprOr{a}{b}}{\mathbb{B}}}
  }
  \and
  \infer*[left=AND]
  {
    \env{\ty{a}{\mathbb{B}}}\\
    \env{\ty{b}{\mathbb{B}}}
  }
  {
    \env{\ty{\exprAnd{a}{b}}{\mathbb{B}}}
  }
  \and
  \infer*[left=GT]
  {
    \env{\ty{a}{T}}\\
    \env{\ty{b}{T}}\\
    \left[ T \in \{ \mathbb{Z}, \mathbb{D} \} \right]
  }
  {
    \env{\ty{\exprGT{a}{b}}{\mathbb{B}}}
  }
  \and
  \infer*[left=LT]
  {
    \env{\ty{a}{T}}\\
    \env{\ty{b}{T}}\\
    \left[ T \in \{ \mathbb{Z}, \mathbb{D} \} \right]
  }
  {
    \env{\ty{\exprLT{a}{b}}{\mathbb{B}}}
  }
  \and
  \infer*[left=EQ]
  {
    \env{\ty{a}{T}}\\
    \env{\ty{b}{T}}\\
    \left[ T \in \{ \mathbb{Z}, \mathbb{B}, \mathbb{D}, \mathbb{S}, \mathbb{L}_t \} \right]
  }
  {
    \env{\ty{\exprEQ{a}{b}}{\mathbb{B}}}
  }
  \and
  \infer*[left=GTE]
  {
    \env{\ty{a}{T}}\\
    \env{\ty{b}{T}}\\
    \left[ T \in \{ \mathbb{Z}, \mathbb{D} \} \right]
  }
  {
    \env{\ty{\exprGTE{a}{b}}{\mathbb{B}}}
  }
  \and
  \infer*[left=LTE]
  {
    \env{\ty{a}{T}}\\
    \env{\ty{b}{T}}\\
    \left[ T \in \{ \mathbb{Z}, \mathbb{D} \} \right]
  }
  {
    \env{\ty{\exprLTE{a}{b}}{\mathbb{B}}}
  }
\end{mathpar}
  \caption{\label{fig:rules}The formal definition of the Typing Rules for \thePolicyLang}
\end{figure}

\Cref{fig:rules} presents the \thePolicyLang typing rules.
These rules dictate what it means for an expression/statement to be well-formed within \thePolicyLang and for any expression/statement in \thePolicyLang, we use the typing rules to construct a derivation that provides proof that the expression/statement is well-typed, that is we can apply each rule and form a derivation tree. If we cannot construct this tree then the expression is ill-typed and syntactically not valid.

The Typing Rules define 5 base cases for the 5 value types \thePolicyLang supports (as defined in \cref{fig:syntax}), meaning instances of the 5 types derive directly to their type. Next, the boolean logical operators OR \& AND are defined as requiring two Boolean parameters that also derive to a Boolean type return. The standard comparison operators \textbf{$>$}, \textbf{$<$}, \textbf{$>=$} \& \textbf{$<=$} all take two parameters of a matching type, where the type may be Integer or Date, and derives to a Boolean response. Lastly, the \textbf{$==$} comparison similarly takes two parameters of a matching type, but where the type may be Integer, Boolean, Date, String or List, and also derives to a Boolean return.


\subsection{Substitution}
\label{subsec:substitution}

\begin{figure}[ht]
  \centering
\begin{align*}
  \subst{\mu}{e}{x}
  &
    \Coloneqq
    \begin{cases}
      e&x\equiv\mu\\
      x&x\not\equiv\mu\\
    \end{cases}
  \\
  \subst{\exprOr{a}{b}}{e}{x}&\Coloneqq\exprOr{\subst{a}{e}{x}}{\subst{b}{e}{x}}\\
  \subst{\exprAnd{a}{b}}{e}{x}&\Coloneqq\exprAnd{\subst{a}{e}{x}}{\subst{b}{e}{x}}\\
  \subst{\exprGT{a}{b}}{e}{x}&\Coloneqq\exprGT{\subst{a}{e}{x}}{\subst{b}{e}{x}}\\
  \subst{\exprLT{a}{b}}{e}{x}&\Coloneqq\exprLT{\subst{a}{e}{x}}{\subst{b}{e}{x}}\\
  \subst{\exprEQ{a}{b}}{e}{x}&\Coloneqq\exprEQ{\subst{a}{e}{x}}{\subst{b}{e}{x}}\\
  \subst{\exprGTE{a}{b}}{e}{x}&\Coloneqq\exprGTE{\subst{a}{e}{x}}{\subst{b}{e}{x}}\\
  \subst{\exprLTE{a}{b}}{e}{x}&\Coloneqq\exprLTE{\subst{a}{e}{x}}{\subst{b}{e}{x}}
\end{align*}
  \caption{\label{fig:subst}The formal definition of the Substitution Rules for \thePolicyLang}
\end{figure}

\Cref{fig:subst} presents a standard set of substitution rules for \thePolicyLang.
These rules describe how we can iterate over our expressions/statements in \thePolicyLang and through recursive calls, swap variables for values.\\
Starting with expressions and variables, we describe when $\mu$ is an expression or variable. Next we describe the resolving of the \textbf{or} \& \textbf{and} logical operators' parameters to expressions and variables, followed by similar descriptions for the standard comparisons \textbf{greaterThan}, \textbf{lessThan}, \textbf{equal}, \textbf{greaterThanEqual} \& \textbf{lessThanEqual}.\\
Through the recursive calls, we eventually reach a state where we have visited all expressions \& statements and we can swap in values in place of all variables by working backwards through the recursive calls.


\subsection{Big Step Semantics}
\label{subsec:semantics}

\begin{figure}[ht]
  \centering
\begin{mathpar}
  \infer*[left=$\mathbb{Z}$]
  {
    \\
  }
  {
    i\Downarrow{}\primed{i}
  }
  \and
  \infer*[left=$\mathbb{B}$]
  {
    \\
  }
  {
    b\Downarrow{}\primed{b}
  }
  \and
  \infer*[left=$\mathbb{D}$]
  {
    \\
  }
  {
    d\Downarrow{}\primed{d}
  }
  \and
  \infer*[left=$\mathbb{S}$]
  {
    \\
  }
  {
    s\Downarrow{}\primed{s}
  }
  \and
  \infer*[left=$\mathbb{L}$]
  {
    \\
  }
  {
    l\Downarrow{}\primed{l}
  }
  \and
  \infer*[left=OR]
  {
    a\Downarrow{}\primed{a}\\
    b\Downarrow{}\primed{b}
  }
  {
    \exprOr{a}{b}\Downarrow\primed{a}\vee\primed{b}
  }
  \and
  \infer*[left=AND]
  {
    a\Downarrow{}\primed{a}\\
    b\Downarrow{}\primed{b}
  }
  {
    \exprAnd{a}{b}\Downarrow\primed{a}\wedge\primed{b}
  }
  \and
  \infer*[left=GT]
  {
    a\Downarrow{}\primed{a}\\
    b\Downarrow{}\primed{b}
  }
  {
    \exprGT{a}{b}\Downarrow\primed{a}>\primed{b}
  }
  \and
  \infer*[left=LT]
  {
    a\Downarrow{}\primed{a}\\
    b\Downarrow{}\primed{b}
  }
  {
    \exprLT{a}{b}\Downarrow\primed{a}<\primed{b}
  }
  \and
  \infer*[left=EQ]
  {
    a\Downarrow{}\primed{a}\\
    b\Downarrow{}\primed{b}
  }
  {
    \exprEQ{a}{b}\Downarrow\primed{a}\equiv\primed{b}
  }
  \and
  \infer*[left=GTE]
  {
    a\Downarrow{}\primed{a}\\
    b\Downarrow{}\primed{b}
  }
  {
    \exprGTE{a}{b}\Downarrow\primed{a}\geq\primed{b}
  }
  \and
  \infer*[left=LTE]
  {
    a\Downarrow{}\primed{a}\\
    b\Downarrow{}\primed{b}
  }
  {
    \exprLTE{a}{b}\Downarrow\primed{a}\leq\primed{b}
  }
  \end{mathpar}
  \caption{\label{fig:semantics}The formal definition of the Big Step Semantics for \thePolicyLang}
\end{figure}

\Cref{fig:semantics} presents the Big-Step semantics for \thePolicyLang, here we use \emph{real} operations to show how an expression is reduced using \emph{real} integer and boolean operators.\\
Operational semantics describe how we evaluate our programs, describing how we can \emph{reduce} the \thePolicyLang language expressions and statements to a single value. Here we present the Big-Step semantics of evaluating expressions to describe the transition from an expression to the final result, skipping over the intermediate computations (as described in \citet{Myers2007}).\\
This intentionally abstracts away the finer details of the operations required to evaluate an individual expression, but still presents the syntactic process. This abstraction was considered beneficial since the expressions available in \thePolicyLang are frequently used across many languages and other resources are available that better describe the Small Step semantics of similar expressions (see \citet{Sewell2009, DBLP:conf/lomaps/Schmidt96}).


\subsection{Interpretation}
\label{subsec:interpretation}

\begin{figure}[ht]
  \centering
\begin{align*}
  \textsc{\thePolicyLang}&\rightarrow\textsc{Python (PyOpenABE)}\\
  \interpB{\TyInt}&\Coloneqq\text{\ttfamily int}\\
  \interpB{\mathbb{B}}&\Coloneqq\text{\ttfamily bool}\\
  \interpB{\mathbb{D}}&\Coloneqq\text{\ttfamily datetime.date}\\
  \interpB{\mathbb{S}}&\Coloneqq\text{\ttfamily str}\\
  \interpB{\mathbb{L}}&\Coloneqq\text{\ttfamily list}\\
  \interpB{n}&\Coloneqq\text{\ttfamily int(}n\text{\ttfamily)}\\
  \interpB{\EnumFalse}&\Coloneqq\text{\ttfamily False}\\
  \interpB{\EnumTrue} &\Coloneqq\text{\ttfamily True}\\
  \interpB{d}&\Coloneqq\text{\ttfamily datetime.date(}d\text{\ttfamily)}\\
  \interpB{s}&\Coloneqq\text{\ttfamily str(}s\text{\ttfamily)}\\
  \interpB{l_s}&\Coloneqq\text{\ttfamily [}\interpB{l}\alt l \leftarrow l_s\text{\ttfamily]}\\
  \interpB{sub}&\Coloneqq\interpB{sub}_{s}\\
  \interpB{proj_{sub}}&\Coloneqq\interpB{proj_{sub}}_{s}\\
  \interpB{env}&\Coloneqq\interpB{env}_{e}\\
  \interpB{proj_{env}}&\Coloneqq\interpB{proj_{env}}_{e}\\
  \interpB{res}&\Coloneqq\interpB{res}_{r}\\
  \interpB{proj_{res}}&\Coloneqq\interpB{proj_{res}}_{r}\\
  \interpB{\exprOr{a}{b}}  &\Coloneqq\interpB{a}\,\text{\ttfamily or}\,\interpB{b}\\
  \interpB{\exprAnd{a}{b}} &\Coloneqq\interpB{a}\,\text{\ttfamily and}\,\interpB{b}\\
  \interpB{\exprGT{a}{b}}  &\Coloneqq\interpB{a}\,\text{\ttfamily \textgreater}\,\interpB{b}\\
  \interpB{\exprLT{a}{b}}  &\Coloneqq\interpB{a}\,\text{\ttfamily \textless}\,\interpB{b}\\
  \interpB{\exprEQ{a}{b}}  &\Coloneqq\interpB{a}\,\text{\ttfamily ==}\,\interpB{b}\\
  \interpB{\exprGTE{a}{b}}  &\Coloneqq\interpB{a}\,\text{\ttfamily \textgreater =}\,\interpB{b}\\
  \interpB{\exprLTE{a}{b}}  &\Coloneqq\interpB{a}\,\text{\ttfamily \textless =}\,\interpB{b}
\end{align*}
  \caption{\label{fig:interp}The formal definition of the Interpretation Rules for \thePolicyLang}
\end{figure}

In this final section we describe how we interpret \thePolicyLang to another form, in this case concrete Python expressions for the \PyOpenABE bindings library (described in \Cref{subsec:bkgr_pyopenabe}), with \cref{fig:interp} presenting these Interpretation rules and like substitution, interpretation recursively operates over each statement and expression in \thePolicyLang. At each step replacing the expression from \thePolicyLang with it's equivalent Python form. As we are interpreting \thePolicyLang we do not provide operational semantics, since Python provides this for us.

