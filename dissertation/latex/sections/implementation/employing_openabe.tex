\section{Building with \OpenABE \& \PyOpenABE}
\label{sec:impl_openabe_libs}

\Cref{sec:bkgr_openabe} describes the \OpenABE library and its implementation of \acrfull{abe}, with \Cref{subsec:bkgr_pyopenabe} describing the \PyOpenABE bindings. The \OpenABE library is implemented with the C language and although the \PyOpenABE bindings are implemented in Python, they use the Cython language to bind Python functions to their C language equivalents (described in \Cref{subsec:bkgr_pyopenabe}).

As such, even though the \acrfull{mks} \& \acrfull{crs} are implemented with Python's Flask microframework and can directly use the \PyOpenABE library, the servers must also have a functioning C environment and the compiled \OpenABE library. The \theResServer system is designed for UNIX systems, meaning that the \acrshort{mks}, \acrshort{prs} \& \acrshort{crs} are all built for running on UNIX \acrshort{os} devices with C and Python environments installed and setup, this is an assumption for running the system.
\vskip 0.5em
Whilst installation instructions are provided individually by all the tools, libraries \& packages used by the \theResServer system, all installations rely on the environment being setup correctly. Throughout the project, this did not prove to be an issue for any of the software used, with the sole exception being the \OpenABE library. As such, a key requirement for running the \acrshort{mks} \& \acrshort{crs} services, is that the environment be prepared with a compiled \& tested version of the \OpenABE library before installation of any part of the \theResServer system (the \acrshort{prs} excluded, as it does not require \OpenABE or \PyOpenABE).

Further, the \PyOpenABE library of bindings must also be compiled \& tested before installation of either the \acrshort{mks} \& \acrshort{crs}, but the \PyOpenABE library appears to compile sufficiently as long as the \OpenABE library is properly configured.
\vskip 0.5em
As the \PyOpenABE library consists of bindings to the \OpenABE library, the values passed to \PyOpenABE functions (shown in \Cref{lst:python_encrypt} \& \Cref{lst:python_decrypt}) are directly passed onto the underlying \OpenABE operations. Requiring that a policy for a resource to be encrypted with must be valid or a fatal error will be returned by \OpenABE.

\begin{lstlisting}[language=python, float, caption={Python code showing the encryption of a file using the \PyOpenABE library.}, label=lst:python_encrypt]
    policy = "username:jspringer or (staff and job_field:Research & Teaching) or (student and student_level = 2 and (enrolled_course:2001 and enrolled_course:2003 and enrolled_course:2007))"
    openabe, cpabe = create_cpabe_instance(MASTER_PUBLIC_KEY)
    try:
        ct_file = cpabe.encrypt(policy, file.read())
    except pyopenabe.PyOpenABEError as err:
        del openabe, cpabe
        flash(f"PyOpenABEError: {err}", 'danger')
        return render_template('encrypt.html', global_attrs=GLOBAL_ABE_ATTRS)
    del openabe, cpabe
\end{lstlisting}

The policy builder described in \Cref{sec:design_pol_build} provides this validation for users, by allowing a user to construct a new policy in \thePolicyLang and have it interpreted straight to a \PyOpenABE (and \OpenABE) compliant form. This interpretation is processed by \acrshort{html} \& Javascript in the user's browser and then sends the interpreted policy to the \acrshort{crs} Flask web server to be encrypted, with \PyOpenABE's \texttt{encrypt} function, as shown in \Cref{lst:python_encrypt}.

Where \texttt{MASTER\_PUBLIC\_KEY} is the binary object representing the master public key, \texttt{file} is a binary stream of the file to be encrypted and \texttt{ct\_file} is a binary stream of the resulting encrypted file. The policy shown is the \PyOpenABE compliant policy as interpreted from the policy building described in \Cref{sec:design_pol_build} for Case Study \#1 (\Cref{fig:case_study_policy_1})

\begin{lstlisting}[language=python, float, caption={Python code showing the decryption of an encrypted file using the \PyOpenABE library.}, label=lst:python_decrypt]
    openabe, cpabe = create_cpabe_instance(MASTER_PUBLIC_KEY)
    cpabe.importUserKey(username, key_bytes)
    try:
        dec_file = cpabe.decrypt(username, file_bytes)
    except pyopenabe.PyOpenABEError as err:
        flash(f"Decryption of file failed: {err}", 'danger')
        dec_file = None
    del openabe, cpabe
\end{lstlisting}

When performing the decryption, the user provides an encrypted file that they wish to decrypt along with their user key, however in the case that the user has logged into the \acrshort{crs} with the pseudo-Authentication Service, the \acrshort{crs} will automatically retrieve the user's private key from the MongoDB database. The encrypted file and user key are then provided to \PyOpenABE's \texttt{decrypt} function, as shown in \Cref{lst:python_decrypt}, which attempts to decrypt the file and if successful, the decrypted file is returned to the user.

Where \texttt{MASTER\_PUBLIC\_KEY} is the binary object representing the master public key, \texttt{key\_bytes} is a binary stream of the user's key, \texttt{username} is a string representing the user's username, \texttt{file\_bytes} is a binary stream of the encrypted file about to be decrypted and \texttt{dec\_file} is a binary stream of the resulting decrypted file.
