\section{Python Web Servers}
\label{sec:impl_web_srvrs}

For the \theResServer system implementation, we require both a \acrfull{mks} and a \acrfull{prs} serving the requirements described in \cref{subsec:analysis_deployment_mks} and \cref{subsec:analysis_deployment_prs}. Both servers are to be implemented as web servers, allowing for simple \acrshort{html} \acrshort{gui}s to be built for user interactions. This allows for non-technical users to use the system without having to learn how to use \acrshort{cli} tools.\\
Additionally, since the \acrshort{mks} would have to integrate the \OpenABE library, it would need to be built with the C language or instead integrate with the \PyOpenABE library of bindings. Since building a web server in the C language was deemed to be an overly complex task for the length of time available to the project, and Python already offers web server frameworks (such as \href{http://flask.pocoo.org/}{Flask} \& \href{https://www.djangoproject.com/}{Django}), the project selected Python as the main language for implementation.\\
As a result, the \acrshort{mks} was built with the Python \href{http://flask.pocoo.org/}{Flask microframework} for its lightweight design and minimal feature set. Ultimately, relying on the fact that both web servers would only require very basic \acrshort{html} \acrshort{gui}s, which would not require the bulk of a Django project. Either framework would have provided the seamless integration of \PyOpenABE that was required for the \acrshort{mks}.
\vskip 0.5em
The \acrshort{prs} did not require integration with the \PyOpenABE library of bindings (as described in \cref{subsec:analysis_deployment_prs}) however did require a connection to a local database for storing metadata of resources. MongoDB was selected for this purpose (as explained in \cref{sec:impl_mongodb}) and Python offers the official \href{https://api.mongodb.com/python/current/}{PyMongo distribution} for this exact connection functionality.\\
Lastly, Python also offers a fuzzy string matcher, the \href{https://pypi.org/project/fuzzywuzzy/}{FuzzyWuzzy} package, to meet the filename searching needs detailed in \cref{sec:design_file_search}. This package combined with the built-in string matching of MongoDB (detailed in \cref{sec:impl_mongodb}) provides both methods required for filename searching.
