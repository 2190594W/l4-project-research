\section{Creating the Client Server}
\label{sec:impl_client_srvr}

With both the \acrshort{mks} \& \acrshort{prs} built using Python's \href{http://flask.pocoo.org/}{Flask microframework}, when it came to building the \acrfull{crs}, the Flask microframework was selected again. This offered a level of code parity between the three products and offered the opportunity to reuse code from both the \acrshort{mks} \& \acrshort{prs} in the \acrfull{crs}, since it required integration with the \PyOpenABE library of bindings (like the \acrshort{prs}) as well as a PyMongo connection to a locally running MongoDB server (like the \acrshort{mks}).

The PyMongo connection was required to offer a pseudo-Authentication Service to the \acrshort{crs} for the project, in place of a real-world integration with a Single-Sign-On service. Whilst the \PyOpenABE library was required to offer encryption \& decryption services to the user, before uploading a resource and after downloading a resource.
\vskip 0.5em
The \acrshort{crs} was built to only run locally on a user's device and the pseudo-Authentication Service was built as a proof-of-concept as well. This means that the service is acknowledged as insecure, since a part of the service relies on storing the user's private user key in plaintext in the \textit{Users} table of the MongoDB database.

Importantly, this key \textbf{never} leaves the user's device during any encryption or decryption tasks, meaning the key is only vulnerable if the user's device \acrshort{os} is directly compromised. This means that in the event of the user's \acrshort{os} being compromised, their user key file would \textit{also} have been compromised anyway, and so the decision to have the user's key in plaintext within the database (\textit{for a proof-of-concept feature}) is not considered high risk. In a real-world deployment of the \theResServer system, this decision would not have to be made, as the system would be integrated with an external Authentication Service such as the university's Single-Sign-On system instead.

The \acrshort{crs} \textbf{does} implement secure password hashing with user accounts created for the pseudo-Authentication Service, using Python's package for \href{https://github.com/p-h-c/phc-winner-argon2}{argon2}, winner of the \href{https://password-hashing.net/}{Password Hashing Competition}, the pip package \href{https://pypi.org/project/argon2-cffi/}{\textit{argon2-cffi}}. This ensures that should a user set a password during registration that they then use with other services and their device becomes compromised, the \acrshort{crs} database cannot be abused to gain access to other services.
