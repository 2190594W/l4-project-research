\section{Risk Assessment}
\label{sec:eval_risk_asses}

The \theResServer system is designed and implemented to be cryptographically secure with careful consideration of the security risks of deploying a public system. Where possible, steps have been taken to mitigate any risks to the system (see \Cref{sec:analysis_security} \& \Cref{sec:impl_web_srvrs}) and if not possible, the deployment or implementation has been designed to limit the system's exposure to risks (see \Cref{subsec:analysis_deployment_mks}).

Identifying the risks to the system was vital to creating a secure system that can be deployed publicly by the \acrfull{dcs} and ensured that the implementation was able to mitigate as many risks as possible. Identifying the risks to a system is not a simple task \citep{Gadd2003} and multiple organisations offer guidance on completing successful risk assessments without the common pitfalls \citep{HSE2014, EuropeanCommission2015}.
\vskip 0.5em
Internationally, the International Organisation of Standards (ISO) publishes standards that are applied across the world and as such its legislature is a more widely recognised body of work than that of the UK's Health and Safety Executive. As such this project decided to follow the ISO 27005 \citep{JointTechnicalCommitteeISOIIECJTC12008} for conducting a risk assessment on the \theResServer system.

Such a risk assessment requires an iterative process for identifying assets and their respective risks:
\begin{enumerate}
  \item
    Identify the assets in a system
  \item
    Identify the threats \& vulnerabilities
  \item
    Identify and estimate the risks to the assets
  \item
    Evaluate each risk
    \begin{itemize}
      \item[]
        If risk can be treated with a fix, jump to item \#5
      \item[]
        If risk can be treated through communication, jump to item \#6
      \item[]
        Risk cannot be treated, jump to item \#7
    \end{itemize}
  \item
    Carry out the fix and jump to item \#2
  \item
    Communicate the risk to users and jump to item \#2
  \item
    Risk cannot be treated, accept risk
  \item
    Continue to perform risk assessments, re-iterate for next assessment, jump to item \#1
\end{enumerate}

This process should be repeated for the life of a system, continually re-assessing the risks as the system evolves or expands, or as the environment changes. In the case of the \theResServer system the assets were identified for the \acrfull{mks} 

\begin{table}[]
  \label{tab:assets_mk}
  \rowcolors{2}{}{gray!3}
  \begin{tabularx}{\linewidth}{lX}
    \textbf{Asset}        & \textbf{Description} \\
    Master Private Key file   &	Extremely vital and dangerous. Whole system relies on this resource staying secret. Signs all keys in use and can sign any arbitrary attributes.  \\
    Master Public Key file    &	Not dangerous, value is distributed as part of normal operation. \\
    Global Attributes file    &	As above, is distributed as part of normal operation but potentially reveals information on the system. \\
    Server Secret (sessions)  &	Secret used to set up sessions with users, and generate CSRF tokens. Potentially dangerous. \\
    Local web server files    &	Contains other config files, but also the key files. Needs protection. \\
    jinja2 plugin             &	Plugin to generate templates. Lowish risk, but an external party provides software. \\
    flask plugin              &	Tool to create and run a web server, potentially damaging. Produced and updated by an external party. \\
    PyOpenABE bindings        &	Python bindings for OpenABE library for all user key generation. Clearly high risk. Also maintained by external party. \\
    cython lib/plugin         &	Python tool to compile python down to C. Interprets all bindings. So as above. \\
    Python3 lib               &	The Python3 library. Similar to above, lower vulnerability as very openly and globally reviewed. \\
    OpenABE C lib             &	OpenABE library for all user key generation. Clearly high risk. Maintained by external party. \\
    C lib                     &	The C library. Similar to Python, low vulnerability as extremely openly and globally reviewed. Slow to update as well. \\
    Firewall                  &	Firewall of the host. Should block incoming requests. May not even be necessary as Key Server should be offline when deployed. \\
    UNIX OS                   &	The UNIX OS the host is running on. Although offline, external party software so potential risk.
  \end{tabularx}
\end{table}

\begin{table}[]
  \label{tab:example_threats_vulns}
  \begin{tabularx}{\linewidth}{cXX}
    \rowcolor[HTML]{8497B0}
    \multicolumn{2}{c}{\cellcolor[HTML]{8497B0}\textbf{Threats}} & \multicolumn{1}{c}{\cellcolor[HTML]{8497B0}} \\
    \rowcolor[HTML]{FFD966}
    \multicolumn{1}{l}{\cellcolor[HTML]{FFD966}\textbf{Threat-Source}} & \textbf{Threat-Actions} & \multicolumn{1}{c}{\multirow{-2}{*}{\cellcolor[HTML]{8497B0}\textbf{Vulnerabilities}}} \\
    \rowcolor[HTML]{A9D08E}
    \cellcolor[HTML]{F3B084}{\color[HTML]{000000} } & Fire & Irreparable fire damage to equipment \\
    \rowcolor[HTML]{A9D08E}
    \cellcolor[HTML]{F3B084}{\color[HTML]{000000} } & Water Damage & Irreparable water damage to equipment \\
    \rowcolor[HTML]{A9D08E}
    \cellcolor[HTML]{F3B084}{\color[HTML]{000000} } & Pollution & Damage from pollution \\
    \rowcolor[HTML]{A9D08E}
    \cellcolor[HTML]{F3B084}{\color[HTML]{000000} } & Major Accident & Physical accident to equipment \\
    \rowcolor[HTML]{A9D08E}
    \cellcolor[HTML]{F3B084}{\color[HTML]{000000} } & \cellcolor[HTML]{A9D08E} & Lack of periodic replacement schemes \\
    \rowcolor[HTML]{A9D08E}
    \cellcolor[HTML]{F3B084}{\color[HTML]{000000} } & \cellcolor[HTML]{A9D08E} & Inadequate recruitment procedures (untrained/unskilled staff) \\
    \rowcolor[HTML]{A9D08E}
    \cellcolor[HTML]{F3B084}{\color[HTML]{000000} } & \multirow{-3}{*}{\cellcolor[HTML]{A9D08E}Destruction of Equipment or Media} & Inadequate or careless use of physical access control to buildings and rooms \\
    \rowcolor[HTML]{A9D08E}
    \multirow{-8}{*}{\cellcolor[HTML]{F3B084}{\color[HTML]{000000} Physical Damage}} & Dust, Corrosion, Freezing & Susceptibility to humidity, dust, soiling
  \end{tabularx}
\end{table}

\begin{table}[]
  \label{tab:example_vulns_risks}
  \begin{tabularx}{\linewidth}{Xllllll}
    \rowcolor[HTML]{9BC1E6}
    \multicolumn{1}{c}{\cellcolor[HTML]{8497B0}} & \multicolumn{3}{l}{\cellcolor[HTML]{9BC1E6}\textbf{Master Private Key file}} & \multicolumn{3}{l}{\cellcolor[HTML]{9BC1E6}\textbf{Master Public Key file}} \\
    \multicolumn{1}{c}{\multirow{-2}{*}{\cellcolor[HTML]{8497B0}\textbf{Vulnerabilities}}} & \cellcolor[HTML]{D87B79}\textbf{Impact} & \cellcolor[HTML]{C6E0B4}\textbf{Likelihood} & \cellcolor[HTML]{8EA9DB}\textbf{Risk} & \cellcolor[HTML]{D87B79}\textbf{Impact} & \cellcolor[HTML]{C6E0B4}\textbf{Likelihood} & \cellcolor[HTML]{8EA9DB}\textbf{Risk} \\
    \cellcolor[HTML]{A9D08E}Irreparable fire damage to equipment & 5 & 3 & \cellcolor[HTML]{FDBB7B}15 & 2 & 3 & \cellcolor[HTML]{A3C37C}6 \\
    \rowcolor[HTML]{EFEFEF}
    \cellcolor[HTML]{A9D08E}Irreparable water damage to equipment & 5 & 3 & \cellcolor[HTML]{FDBB7B}15 & 2 & 3 & \cellcolor[HTML]{A3C37C}6 \\
    \cellcolor[HTML]{A9D08E}Damage from pollution & 5 & 1 & \cellcolor[HTML]{96C27C}5 & 2 & 1 & \cellcolor[HTML]{6FBF7B}2 \\
    \rowcolor[HTML]{EFEFEF}
    \cellcolor[HTML]{A9D08E}Physical accident to equipment & 4 & 2 & \cellcolor[HTML]{BCC57C}8 & 2 & 2 & \cellcolor[HTML]{88C17B}4 \\
    \cellcolor[HTML]{A9D08E}Lack of periodic replacement schemes & 1 & 3 & \cellcolor[HTML]{7CC07B}3 & 1 & 3 & \cellcolor[HTML]{7CC07B}3 \\
    \rowcolor[HTML]{EFEFEF}
    \cellcolor[HTML]{A9D08E}Inadequate recruitment procedures (untrained/unskilled staff) & 3 & 4 & \cellcolor[HTML]{F0C97D}12 & 2 & 3 & \cellcolor[HTML]{A3C37C}6 \\
    \cellcolor[HTML]{A9D08E}Inadequate or careless use of physical access control to buildings and rooms & 5 & 2 & \cellcolor[HTML]{D6C77D}10 & 2 & 3 & \cellcolor[HTML]{A3C37C}6 \\
    \rowcolor[HTML]{EFEFEF}
    \cellcolor[HTML]{A9D08E}Susceptibility to humidity, dust, soiling & 3 & 2 & \cellcolor[HTML]{A3C37C}6 & 2 & 3 & \cellcolor[HTML]{A3C37C}6
  \end{tabularx}
\end{table}
