\section{Security Considerations}
\label{sec:analysis_security}

A Resource Server must be built with security as a top priority, or risk resources being vulnerable to attacks and security breaches. Access to resources \textit{can} be managed with Access Control alone, ensuring that only authorised users may view or download a particular resource and whilst this \textit{can} meet the basic security needs for an organisation, no fallback protection is provided in the event of a security breach.

\subsection{Securing Resources}
\label{subsec:analysis_sec_res}

Although open and public resource repositories do exist, such as the \href{https://commons.wikimedia.org/wiki/Main_Page}{Wikimedia Commons} service, they are the edge case across the Internet. A more common setup is a public service with both public and private resources, such as GitHub where code repositories may be set to `private' if a user wishes their code to be hidden from the general public.
\vskip 0.5em
Access Control is enough protection for many services, as a user must authenticate to gain access to different routes and locations of a website, with their account dictating if access should be granted or denied. The implications and intricacies of such an authentication system are not within the scope of this project, however the following may provide further information \textemdash\ \citet{Sandhu1996, Johnston2004, Fu2001}.

Employment of a properly configured Access Control service works perfectly, until we consider the possibility of a system breach occurring. In such a scenario, the attacker may have broken or circumvented the authentication or authorisation system(s) and gained access to the system's storage. In such a situation, the unencrypted resources are completely vulnerable to the breaching attacker and any restrictions defined by the Access Control system are rendered worthless.
\vskip 0.5em
Due to these risks, a system should offer a method of protecting stored resources, even when a breach has occurred. In the case of password storage, \citet{Teat2011} show that the risk of breaches is mitigated with cryptographic hashing, however since hashing is a one-way operation we cannot use it for resource storage.

Instead we must integrate a form of at-rest encryption to securely store resources as unintelligible ciphertexts. Such encryption relies on the employment of a block cipher algorithm and the current NIST guidance offers two approved block cipher algorithms \citep{NIST2017}; \acrfull{aes} and Triple DES as described by \citet{Daemen2003} \& \citet{Barker2017}.

Both block cipher algorithms are forms of \textit{symmetric}-key encryption (\Cref{sec:bkgr_encryption}) which typically requires the service provider to store both the decryption key(s) and the resources. This ultimately leaves a single party in control of both keys and encrypted resources, as with Google Drive \& OneDrive business accounts \citep{Winder2018}.

\subsection{Choosing ABE}
\label{subsec:analysis_abe}

\acrfull{abe} by comparison offers the possibility of storing encrypted resources with block cipher algorithms such as \acrshort{aes} 128-bit, without the requirement of storing keys. A \textit{ciphertext policy} \acrshort{abe} system, for example, binds an \acrshort{aes} ciphertext with a policy that describes whom may decrypt it \citep{Akinyele2011} and employs Public Key Cryptography ( \Cref{sec:bkgr_pub_key_infr}) to enforce the policy.
\vskip 0.5em
Since an \acrshort{abe} system relies on Public Key Cryptography, anyone may encrypt a resource with a policy using the distributed master public key, however only private user keys are capable of decrypting resources.
These keys are generated from a user's attributes, as defined by the organisation, such as \texttt{(role$=$Staff $\alt$ department$=$\acrshort{dcs} $\alt$ jobField$=$Research \& Teaching)}.

These keys \textbf{must} be generated and then signed by a designated \acrfull{mks} \textit{(or private key generator)} that remains the only entity with access to the master \textit{private} key. When a user key is created they are created with a random seed, ensuring that two users with different levels of access cannot collude to form a new key with more access than they have individually \citep{Akinyele2011}, providing implicit \textit{collusion resistance}.

\subsection{ABE Integration}
\label{subsec:analysis_abe_impl}

To work securely \& efficiently, an \acrshort{abe} system must implement Public Key Cryptography as a layer \textit{on top of} a block cipher algorithm such as \acrshort{aes}. Although Public Key Cryptography \textemdash\ for example the RSA algorithm \citep{Barker2016} \textemdash\ can encrypt whole resources without \acrshort{aes}, the process is much less efficient and consumes a greater quantity of compute resources in the process \citep{AlHasib2008}.
\vskip 0.5em
An \acrshort{abe} system first uses \acrshort{aes} to encrypt a resource's contents, producing an \acrshort{aes} \textit{symmetric} key and a ciphertext, then the \acrshort{abe} system uses Public Key Cryptography to encrypt the key with a Boolean formula policy.

This Boolean policy has now been mathematically bound to the \acrshort{aes} ciphertext with Public Key cryptography, such that it is an absolute requirement for decryption, that the formula resolve correctly for a provided decryption key \citep{Sahai2005}.
\vskip 0.5em
For the decryption process, the system then attempts to resolve the policy with the user's attributes. If, and only if, the policy resolves to true, the \acrshort{aes} key will be decrypted by the user key. The system then proceeds to execute \acrshort{aes} decryption on the ciphertext \citep{Akinyele2011}, finally returning an unencrypted resource.
