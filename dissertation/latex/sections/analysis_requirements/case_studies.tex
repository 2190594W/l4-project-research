\section{Case Studies}
\label{sec:analysis_case_studies}

From the work completed in \cref{ch:background} and sections \ref{sec:analysis_security}, \ref{sec:analysis_deployment} \& \ref{sec:analysis_enrolment}, the project had determined the users of the \theResServer system (see \cref{appendix:roles_users}) as well as the features the service would need to provide. With this information, several Case Studies were designed to show the extensibility of the system in real-world scenarios. Of the six Case Studies presented here, \cref{subsec:analysis_case_studies_3} and \cref{subsec:analysis_case_studies_4} are examined and discussed at the most in-depth level.

\subsection{\#0 \textemdash\ Past Exam Paper}
\label{subsec:analysis_case_studies_0}

Case Study \#0 considers a past exam paper uploaded as an encrypted resource by a member of the Administration staff for an arbitrary course. The staff member would like the past paper to be shared with all members of staff and students in the Department of Computing Science.
\vskip 0.5em
For this study the following details have been assumed:
\begin{itemize}
  \item
    the solutions were encrypted and then uploaded by the Admin staff member
  \item
    the staff member is Teresa Bonner with username `tbonner'
\end{itemize}
\vskip 0.5em
With the details extracted, we can determine the attributes for the policy, where we define \textbf{\textit{Subject} s} and \textbf{\textit{Environment} e} as:
\begin{itemize}
  \item[]
    \textbf{s} $\Rightarrow$ role, accountActiveUntil, username
  \item[]
    \textbf{e} $\Rightarrow$ currentDate
\end{itemize}

Applying the identified attributes for the scenario provides the policy in \cref{fig:case_study_policy_0} which would have been embedded in the encrypted lecture slides by the staff member before upload. The encrypted past paper can then remain stored on the server but only accessible to staff members and students.

\begin{figure}[ht]
  \centering
\begin{align*}
  \text{Policy(\textbf{s},\textbf{e})}
  &
    \leftarrow
    \text{username(\textbf{s})} \equiv \text{`tbonner'}
  \\
  &
    \phantom{::}\vee
    \text{role(\textbf{s})} \equiv \text{`Staff'} \mid \text{`Student'}
  \\
  &
    \phantom{::}\wedge
    \text{accountActiveUntil(\textbf{s})} \geq \text{currentDate(\textbf{e})}
\end{align*}
  \caption{
    \label{fig:case_study_policy_0}
    Case Study \#0 policy dictating access to a past exam paper.
    Successful decryption would be possible for the author of the slides (with username `tbonner') \textbf{or} any member of staff \textbf{or} any student. In all cases an active account is also required.
  }
\end{figure}

\subsection{\#1 \textemdash\ Course Lecture Slides}
\label{subsec:analysis_case_studies_1}

Case Study \#1 considers a set of Lecture Slides uploaded as an encrypted resource by the Lecturer of an arbitrary course. The Lecturer would like the slides to be shared with other members of staff, to allow for suggestions and even collaboration, but only for Research \& Teaching staff. The Lecturer does not want the slides to be accessible to students until 24 hours before the scheduled lecture date, inline with university policy. For the sake of the scenario, the Java Programming 2 course was selected with Course Code 2001 to show the creation of a policy.
\vskip 0.5em
For this study the following details have been assumed:
\begin{itemize}
  \item
    the lecture slides have been uploaded in advance of the lecture date
  \item
    the lecture date is scheduled for 23/10/19
  \item
    the solutions were encrypted and then uploaded by the Lecturer
  \item
    the Lecturer for 2001 is Jeremy Springer with username `jspringer'
  \item
    Course 2001 is a Level 2 course
\end{itemize}
\vskip 0.5em
With the details extracted, we can determine the attributes for the policy, where we define \textbf{\textit{Subject} s} and \textbf{\textit{Environment} e} as:
\begin{itemize}
  \item[]
    \textbf{s} $\Rightarrow$ role, jobField, studentLevel, enrolledCourses, accountActiveUntil, username
  \item[]
    \textbf{e} $\Rightarrow$ currentDate
\end{itemize}

Applying the identified attributes for the scenario provides the policy in \cref{fig:case_study_policy_1} which would have been embedded in the encrypted lecture slides by the Lecturer before upload. The encrypted slides can then remain stored on the server but only accessible to Research \& Teaching staff members until 24 hours before the lecture date of 23/10/19, when students on the course will also be granted access.

\begin{figure}[ht]
  \centering
\begin{align*}
  \text{Policy(\textbf{s},\textbf{e})}
  &
    \leftarrow
    \text{username(\textbf{s})} \equiv \text{`jspringer'}
  \\
  &
    \phantom{::}\vee
    \text{( role(\textbf{s})} \equiv \text{`Staff'}
  \\
  &
    \phantom{::::::::}\wedge
    \text{jobField(\textbf{s})} \equiv \text{`Research \& Teaching' )}
  \\
  &
    \phantom{::}\vee
    \text{( role(\textbf{s})} \equiv \text{`Student'}
  \\
  &
    \phantom{::::::::}\wedge
    \text{studentLevel(\textbf{s})} \equiv \text{2}
  \\
  &
    \phantom{::::::::}\wedge
    \text{enrolledCourses(\textbf{s})} \equiv \text{2001}
  \\
  &
    \phantom{::::::::}\wedge
    \text{currentDate(\textbf{e})} \geq \text{22 October 2019 )}
  \\
  &
    \phantom{::}\wedge
    \text{accountActiveUntil(\textbf{s})} \geq \text{currentDate(\textbf{e})}
\end{align*}
  \caption{
    \label{fig:case_study_policy_1}
    Case Study \#1 policy dictating access to a set of course 2001 lecture slides.
    Successful decryption would be possible for the author of the slides (with username `jspringer') \textbf{or} a member of staff in the Research \& Teaching field \textbf{or} a Level 2 student enrolled in the 2001 course if the lecture slides' release date has passed. In all cases an active account is also required.
  }
\end{figure}

\subsection{\#2 \textemdash\ Course Lab Solutions}
\label{subsec:analysis_case_studies_2}

Case Study \#2 considers a lab solution uploaded as an encrypted resource by the Lecturer of an arbitrary course. The Lecturer would like to upload the solutions in advance of the lab date, allowing for other members of staff to verify the solutions, but only for Research \& Teaching staff. The Lecturer does not want the solutions to be accessible to students until 3 days after the scheduled lab date, allowing the Lecturer to talk through the solutions in a lecture before release. The Lecturer would like the solutions to be available to tutors \& demonstrators that will be assisting in the lab, so that they can properly offer help to students.\\
For the sake of the scenario, the 1P Programming course was selected with Course Code 1001 as it offers the added complexity of tutors \& demonstrators, but it should be remembered that any course with labs could have been chosen.
\vskip 0.5em
For this study the following details have been assumed:
\begin{itemize}
  \item
    the lab solutions have been uploaded in advance of the actual lab date
  \item
    the lab date is scheduled for 04/12/19
  \item
    the solutions were encrypted and then uploaded by the Lecturer
  \item
    the Lecturer for 1001 is John Williamson with username `jwilliamson'
  \item
    Course 1001 is a Level 1 course
  \item
    Course 1001 has a sister course, 1017
  \item
    the 1P course labs are assisted by tutors \& demonstrators from Level 4+
\end{itemize}
\vskip 0.5em
With the details extracted, we can determine the attributes for the policy, where we define \textbf{\textit{Subject} s} and \textbf{\textit{Environment} e} as:
\begin{itemize}
  \item[]
    \textbf{s} $\Rightarrow$ role, jobField, studentLevel, enrolledCourses, accountActiveUntil, startDate, endDate, username, studentRole, demonstratorCourses
  \item[]
    \textbf{e} $\Rightarrow$ currentDate
\end{itemize}

Applying the identified attributes for the scenario provides the policy in \cref{fig:case_study_policy_2} which would have been embedded in the encrypted lab solution by the Lecturer before upload. The encrypted lab solution can then remain stored on the server but only accessible to Research \& Teaching staff members and valid tutors \& demonstrators until 3 days after the lab date of 04/12/19, when students on the course will also be granted access.

\begin{figure}[ht]
  \centering
\begin{align*}
  \text{Policy(\textbf{s},\textbf{e})}
  &
    \leftarrow
    \text{username(\textbf{s})} \equiv \text{`jwilliamson'}
  \\
  &
    \phantom{::}\vee
    \text{( role(\textbf{s})} \equiv \text{`Staff'}
  \\
  &
    \phantom{::::::::}\wedge
    \text{jobField(\textbf{s})} \equiv \text{`Research \& Teaching' )}
  \\
  &
    \phantom{::}\vee
    \text{( role(\textbf{s})} \equiv \text{`Student'}
  \\
  &
    \phantom{::::::::}\wedge
    \text{studentLevel(\textbf{s})} \equiv \text{1}
  \\
  &
    \phantom{::::::::}\wedge
    \text{enrolledCourses(\textbf{s})} \equiv \text{[1001, 1017]}
  \\
  &
    \phantom{::::::::}\wedge
    \text{currentDate(\textbf{e})} \geq \text{7 December 2019 )}
  \\
  &
    \phantom{::}\vee
    \text{( role(\textbf{s})} \equiv \text{`Student'}
  \\
  &
    \phantom{::::::::}\wedge
    \text{studentLevel(\textbf{s})} \geq \text{4}
  \\
  &
    \phantom{::::::::}\wedge
    \text{studentRole(\textbf{s})} \equiv \text{`Demonstrator UG'} \mid \text{`Demonstrator PG'} \mid \text{`Tutor'}
  \\
  &
    \phantom{::::::::}\wedge
    \text{demonstratorCourses(\textbf{s})} \equiv \text{[1001, 1017]}
  \\
  &
    \phantom{::::::::}\wedge
    \text{startDate(\textbf{e})} \leq \text{currentDate(\textbf{e})}
  \\
  &
    \phantom{::::::::}\wedge
    \text{endDate(\textbf{e})} \geq \text{currentDate(\textbf{e}) )}
  \\
  &
    \phantom{::}\wedge
    \text{accountActiveUntil(\textbf{s})} \geq \text{currentDate(\textbf{e})}
\end{align*}
  \caption{
    \label{fig:case_study_policy_2}
    Case Study \#2 policy dictating access to a set of course `1001' lab solutions.
    Successful decryption would be possible for the author of the solutions (with username `jwilliamson') \textbf{or} a member of staff in the `Research \& Teaching' field \textbf{or} a Level 2 student enrolled in the `1001' \& `1017' courses if the lab date has passed \textbf{or} a Level 4+ student employed as a demonstrator or tutor that has been assigned to the `1001' \& `1017' course and it is within their dates of employment. In all cases an active account is also required.
  }
\end{figure}

\subsection{\#3 \textemdash\ In-progress Exam Script}
\label{subsec:analysis_case_studies_3}

Case Study \#3 considers an in-progress exam script uploaded as an encrypted resource by the Lecturer of an arbitrary course. The exam script is still a draft and is still being worked on by other members of staff in preparation for the exam period. The Lecturer would like to upload the script in advance of the exam date, allowing for other members of staff to verify the script and offer feedback, but only for Research \& Teaching staff.\\
The script is considered confidential and should only accessible to staff within the Lecturer's Research Group (e.g. FATA, GLASS, IDA) and more specifically their Theme\slash Topic group (e.g. FDTST, MOG, HUSH), unless the Lecturer has specifically granted an individual access. The exam script must also be accessible to Admin staff so that it can be passed to the Exam board for final review. The exam has a scheduled date and a following marking period in which markers will also need access.\\
For the sake of the scenario, the Programming Languages course was selected with Course Code 4016 to show the creation of a policy.
\vskip 0.5em
For this study the following details have been assumed:
\begin{itemize}
  \item
    the exam script is a work in progress resource
  \item
    the exam script has been uploaded in advance of the exam date
  \item
    the solutions were encrypted and then uploaded by the Lecturer
  \item
    the Lecturer's Research Group is FATA
  \item
    the Lecturer's Research Theme\slash Topic is Programming Languages
  \item
    the exam date is 12/05/20
  \item
    the marking deadline is 10/06/20
  \item
    the Lecturer for 4016 is Ornela Dardha with username `odardha'
  \item
    collaborators include Simon Gay (`sgay'), John O'Donnell (`jodonnell')
\end{itemize}
\vskip 0.5em
With the details extracted, we can determine the attributes for the policy, where we define \textbf{\textit{Subject} s} and \textbf{\textit{Environment} e} as:
\begin{itemize}
  \item[]
    \textbf{s} $\Rightarrow$ role, jobField, jobRole, researchGroup, researchTheme, accountActiveUntil, staffRole, markerFrom, markerTo, markerCourses, username
  \item[]
    \textbf{e} $\Rightarrow$ currentDate
\end{itemize}

Applying the identified attributes for the scenario provides the policy in \cref{fig:case_study_policy_3} which would have been embedded in the encrypted exam script by the Lecturer before upload. The encrypted exam script can then remain stored on the server but would only be the Lecturer themselves, as well as the specified collaborators.\\
Further access would be granted to staff members in the Professional, Administrative \& Support field that specifically have the Administration role, as well as Research \& Teaching staff members that belong to the FATA research group and the Programming Languages research theme. Lastly, access would be granted to markers that have been assigned to the 4016 course and will be actively marking during the marking period (12/05/20\textemdash10/06/20).

\begin{figure}[ht]
  \centering
\begin{align*}
  \text{Policy(\textbf{s},\textbf{e})}
  &
    \leftarrow
    \text{username(\textbf{s})} \equiv \text{`odardha'} \mid \text{`sgay'} \mid \text{`jodonnell'}
  \\
  &
    \phantom{::}\vee
    \text{( role(\textbf{s})} \equiv \text{`Staff'}
  \\
  &
    \phantom{::::::::}\wedge
    \text{jobField(\textbf{s})} \equiv \text{`Professional, Administrative \& Support'}
  \\
  &
    \phantom{::::::::}\wedge
    \text{jobRole(\textbf{s})} \equiv \text{`Administration' )}
  \\
  &
    \phantom{::}\vee
    \text{( role(\textbf{s})} \equiv \text{`Staff'}
  \\
  &
    \phantom{::::::::}\wedge
    \text{jobField(\textbf{s})} \equiv \text{`Research \& Teaching'}
  \\
  &
    \phantom{::::::::}\wedge
    \text{researchGroup(\textbf{s})} \equiv \text{`FATA'}
  \\
  &
    \phantom{::::::::}\wedge
    \text{researchTheme(\textbf{s})} \equiv \text{`Programming Languages' )}
  \\
  &
    \phantom{::}\vee
    \text{( role(\textbf{s})} \equiv \text{`Staff'}
  \\
  &
    \phantom{::::::::}\wedge
    \text{staffRole(\textbf{s})} \equiv \text{`Marker'}
  \\
  &
    \phantom{::::::::}\wedge
    \text{markerFrom(\textbf{s})} \geq \text{12 May 2020}
  \\
  &
    \phantom{::::::::}\wedge
    \text{markerTo(\textbf{s})} \leq \text{10 June 2020}
  \\
  &
    \phantom{::::::::}\wedge
    \text{markerCourses(\textbf{s})} \equiv \text{4016}
  \\
  &
    \phantom{::}\wedge
    \text{accountActiveUntil(\textbf{s})} \geq \text{currentDate(\textbf{e})}
\end{align*}
  \caption{
    \label{fig:case_study_policy_3}
    Case Study \#3 policy dictating access to a work-in-progress exam script for course `4016'.
    Successful decryption would be possible for the author of the solutions (with username `odardha') as well as the named collaborators (with usernames `sgay' \& `jodonnell') \textbf{or} for a member of staff in the `Professional, Administrative \& Support' field with the `Administration' role \textbf{or} for a member of staff in the `Research \& Teaching' field that is also in the `FATA' research group and the `Programming Languages' research theme \textbf{or} for a member of staff with the `Marker' role that is assigned to marking duties within the marking period and has been assigned to the `4016' course. In all cases an active account is also required.
  }
\end{figure}

\subsection{\#4 \textemdash\ Level 4 Honours Individual Project}
\label{subsec:analysis_case_studies_4}

Case Study \#4 considers a completed individual project from a Level 4 student, particularly the dissertation resource although the project source code could also be treated identically if submitted as a compressed .tar.gz or .zip file.\\
In this case, the student has encrypted and uploaded their dissertation to the \theResServer system with access to be granted for their supervisor, the assigned reader and additionally, the Project Coordinator.
\vskip 0.5em
For this study the following details have been assumed:
\begin{itemize}
  \item
    the student has the ID and username `2123456z'
  \item
    their supervisor is Quintin Cutts with username `qcutts'
  \item
    their assigned reader is Paul Siebert with username `psiebert'
  \item
    the Project Coordinator is John Williamson with username `jwilliamson'
\end{itemize}
\vskip 0.5em
With the details extracted, we can determine the attributes for the policy, where we define \textbf{\textit{Subject} s} and \textbf{\textit{Environment} e} as:
\begin{itemize}
  \item[]
    \textbf{s} $\Rightarrow$ role, accountActiveUntil, username
  \item[]
    \textbf{e} $\Rightarrow$ currentDate
\end{itemize}

Applying the identified attributes for the scenario provides the policy in \cref{fig:case_study_policy_4} which would have been embedded in the encrypted dissertation by the student before upload. The encrypted dissertation can then remain stored on the server but only accessible to the student and required staff members.

\begin{figure}[ht]
  \centering
\begin{align*}
  \text{Policy(\textbf{s},\textbf{e})}
  &
    \leftarrow
    \text{( role(\textbf{s})} \equiv \text{`Student'}
  \\
  &
    \phantom{::::::}\wedge
    \text{username(\textbf{s})} \equiv \text{`2123456z' )}
  \\
  &
    \phantom{::}\vee
    \text{( role(\textbf{s})} \equiv \text{`Staff'}
  \\
  &
    \phantom{::::::}\wedge
    \text{username(\textbf{s})} \equiv \text{`qcutts'} \mid \text{`psiebert'} \mid \text{`jwilliamson' )}
  \\
  &
    \phantom{::}\wedge
    \text{accountActiveUntil(\textbf{s})} \geq \text{currentDate(\textbf{e})}
\end{align*}
  \caption{
    \label{fig:case_study_policy_4}
    Case Study \#4 policy dictating access to a completed dissertation.
    Successful decryption would be possible for the author of the dissertation (with username `2123456z') \textbf{or} for the three identified staff members (with usernames `qcutts', `psiebert', `jwilliamson'). In all cases an active account is also required.
  }
\end{figure}


\subsection{\#5 \textemdash\ Class Representative Meeting Minutes}
\label{subsec:analysis_case_studies_5}

Case Study \#5 considers the secure storage of the minutes from a Class Representative meeting, which would have been uploaded as an encrypted resource by a member of Admin staff after recording. The minutes should be accessible to all Research \& Teaching staff as well as the rest of the Administration team. Further, all active class reps should also have access to the minutes, as it is assumed that they will have attended the meeting. Lastly, the Head of the \acrshort{dcs} (Head of School) should also have access to the minutes to see what was discussed \& decided.\\
In this case, the member of staff has encrypted and uploaded the minutes to the \theResServer system with access granted as above.
\vskip 0.5em
For this study the following details have been assumed:
\begin{itemize}
  \item
    the member of staff is Tania Galabova with username `tgalabova'
  \item
    the meeting occurred on 12/09/19
\end{itemize}
\vskip 0.5em
With the details extracted, we can determine the attributes for the policy, where we define \textbf{\textit{Subject} s} and \textbf{\textit{Environment} e} as:
\begin{itemize}
  \item[]
    \textbf{s} $\Rightarrow$ role, accountActiveUntil, username
  \item[]
    \textbf{e} $\Rightarrow$ currentDate
\end{itemize}

Applying the identified attributes for the scenario provides the policy in \cref{fig:case_study_policy_5} which would have been embedded in the encrypted minutes by the member of staff before upload. The encrypted minutes can then remain stored on the server but will only be accessible to Research \& Teaching staff members, Administration staff members and the current Head of School. Further access would also be granted to all students presently serving as Class Representatives.

\begin{figure}[ht]
  \centering
\begin{align*}
  \text{Policy(\textbf{s},\textbf{e})}
  &
    \leftarrow
    \text{username(\textbf{s})} \equiv \text{`tgalabova'}
  \\
  &
    \phantom{::}\vee
    \text{( role(\textbf{s})} \equiv \text{`Staff'}
  \\
  &
    \phantom{::::::::}\wedge
    \text{jobField(\textbf{s})} \equiv \text{`Research \& Teaching' )}
  \\
  &
    \phantom{::}\vee
    \text{( role(\textbf{s})} \equiv \text{`Staff'}
  \\
  &
    \phantom{::::::::}\wedge
    \text{jobField(\textbf{s})} \equiv \text{`Professional, Administrative \& Support'}
  \\
  &
    \phantom{::::::::}\wedge
    \text{jobRole(\textbf{s})} \equiv \text{`Administration' )}
  \\
  &
    \phantom{::}\vee
    \text{( role(\textbf{s})} \equiv \text{`Staff'}
  \\
  &
    \phantom{::::::::}\wedge
    \text{staffRole(\textbf{s})} \equiv \text{`Head of School'}
  \\
  &
    \phantom{::::::::}\wedge
    \text{headSchoolFrom(\textbf{s})} \geq \text{17 Sep 2019}
  \\
  &
    \phantom{::::::::}\wedge
    \text{headSchoolTo(\textbf{s})} \leq \text{16 Sep 2020 )}
  \\
  &
    \phantom{::}\vee
    \text{( role(\textbf{s})} \equiv \text{`Student'}
  \\
  &
    \phantom{::::::::}\wedge
    \text{studentRole(\textbf{s})} \equiv \text{`Class Representative'}
  \\
  &
    \phantom{::::::::}\wedge
    \text{classRepFrom(\textbf{s})} \geq \text{17 Sep 2019}
  \\
  &
    \phantom{::::::::}\wedge
    \text{classRepTo(\textbf{s})} \leq \text{16 Sep 2020 )}
  \\
  &
    \phantom{::}\wedge
    \text{accountActiveUntil(\textbf{s})} \geq \text{currentDate(\textbf{e})}
\end{align*}
  \caption{
    \label{fig:case_study_policy_5}
    Case Study \#5 policy dictating access to a Class Rep meeting minutes.
    Successful decryption would be possible for the author of the minutes (with username `tgalabova') \textbf{or} for any `Research \& Teaching' staff members \textbf{or} for any `Administration' staff members \textbf{or} for the `Head of School', if their serving term matches the academic year of the minutes \textbf{or} for any students serving as a `Class Representative' in the academic year of the minutes. In all cases an active account is also required.
  }
\end{figure}
