\begin{abstract}
    Sharing resources securely across organisations or departments is a difficult and daunting task. Common methods rely on forms of Role-based access control (RBAC), however these cannot provide a fine-grained system for precise or modular access.\\
    This problem can be solved with the use of Attribute-based access control (ABAC) instead. Wherein, users are granted specific attributes instead of roles, allowing for any user to be given a precise and even unique set of attributes.
    Use of an Attribute-based encryption (ABE) system further improves security by handling the secure storage and transmission of resources through advanced encryption. Like ABAC, the ABE system utilises policies of attributes for users to determine if a resource should be decrypted by a given user.
    \vskip 0.5em
    This project aims to develop a full resource server product that meets the definition of "cryptographically secure". The product would be built from 2 servers: an online 'dumb' resource server to store the ABE encrypted resources and an offline 'cold storage' master key server.
    An end user would also have a locally running web client 'server' for communication with the resource server, allowing for downloading \& decryption of files, along with encryption \& uploading of their own files.
    \vskip 0.5em
    ``XYZ is bad. This project investigated ABC to determine if it was better.
    ABC used XXX and YYY to implement ZZZ. This is particularly interesting as XXX and YYY have
    never been used together. It was found that
    ABC was 20\% better than XYZ, though it caused rabies in half of subjects.''
\end{abstract}
