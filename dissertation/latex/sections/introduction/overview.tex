Sharing resources securely across organisations or departments is both a difficult and daunting task, with common methods relying on forms of Role-based access control (RBAC) to grant access to, often unencrypted, resource buckets. However these methods are unable to provide a fine-grained system for either precise or modular access and cannot protect the resources should the system become compromised.\\
Clearly, a level of granularity is required in order to configure appropriate access restrictions to resources and provide long-term support for the system through dynamic access control. RBAC is also complex to configure properly and due to the nature of roles, can risk accidental access grants through roles that are too expansive in permissions.\\
These problems can be solved with the use of an Attribute-Based Encryption (ABE) system which improves security by handling the advanced encryption of resources through a defined policy language which allows for unique, granular access to resources on a per-user basis.\\
Unlike RBAC, an ABE system utilises a policy language to create bespoke, per-resource attribute policies that define specific access restrictions for the resource they are embedded into. This means that a user can only decrypt a resource if they can cryptographically prove assignment of the required attributes in their private keys - generated \& signed by a central master key server.

\section{Overview}
\label{sec:intro_overview}

This project aimed to develop a full resource server product that meets the definition of "cryptographically secure"; where resources are encrypted at rest and only transmitted \& stored in a secure, unreadable ciphertext format. Thus, the system would never be aware of the contents of the resources stored and implicitly, the resources will remain secure even in the event of the system becoming compromised. The product as a whole was determined to need 2 servers:
\begin{enumerate}
  \item an offline 'cold storage' master key server to generate user keys
  \item an online 'dumb' resource server to store and manage the ABE encrypted resources
\end{enumerate}
Due to the utilisation of ABE, communication with the resource server is too complex and verbose for a standard user and as such, a tool is provided for users that simplifies interactions. Subsequently, a user can choose to run a simple local web client for this communication with the resource server, providing the following services:
\begin{enumerate}
  \item downloading of encrypted resources
  \item decryption of encrypted resources
  \item encryption of unencrypted resources
  \item uploading of encrypted resources
  \item searching for uploaded resources
  \item determining policies of uploaded resources
  \item creating a policy for a new resource
\end{enumerate}
All these services are then accessible through a basic GUI that aims to obscure the complexity of ABE from the user; reducing the learning curve to use of the product.
