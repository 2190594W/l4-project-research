\section{Aims}
\label{sec:intro_aims}

The main aim of the project was to produce an end product that can serve as a cryptographically secure resource server for a department and its users, with a configurable and dynamic system for access control.\\
Production of the service required careful research and design before implementation to determine both the end users of the service and the needs of the department. To achieve this, the project aimed to identify the users in the scenario of the Department of Computing Science (DCS) with an analysis of the structure of staff and students.
\vskip 0.5em
The project needed to consider the on-boarding process of bringing new users into the system and create a valid procedure for creating cryptographic keys for users. This includes determining the validity of a user's attributes and identifying an authority that can securely be tasked with performing said validation.\\
The service would have to implement and show the security of an ABE system with real-world use cases to prove its effectiveness in the scope of securely distributing resources amongst members of a department. Further, the system would also have to undergo a risk assessment to determine the actual security of the system whilst identifying risk factors within the service.
\vskip 0.5em
Since the project would implement an ABE system and it was determined that creating an ABE library was beyond the defined scope, the project also had to identify and then employ an external ABE library. Determination of which would focus on the extensibility of the library as well as evidence of the library in use for a similar scenario as a departmental resource server.\\
The project looked into the Johns Hopkins Hospital deployment of an electronic medical records system, \citet{Akinyele2011}. This represented a good scenario for comparison, as the secure distribution of medical records amongst staff and patients relied on dynamic and extremely granular access control but with an even higher requirement for security than that of a departmental resource server.
\vskip 0.5em
For the most part, the setup used for the Johns Hopkins deployment would have translated well to the project's deployment scenario, including the on-boarding process by which a user receives their private key from a central admin service and the granularity of the policies used to encrypt records.\\
However, the Johns Hopkins team required the use of mobile applications as end devices and also a need to update the user's private key remotely. Both features that were determined to be very high risk and ultimately unnecessary for the scope of the resource server. Additionally, the Johns Hopkins deployment relied on all new data coming from one source \textemdash\ the hospital \textemdash\ and so was built on the basis that one system would be able to encrypt all records. Aresource server however, must be able to receive resources from many different sources and in the project's case, any member of the DCS would need to be able to encrypt \& decrypt resources locally.
