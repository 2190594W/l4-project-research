\section{Aims}
\label{sec:intro_aims}

The main aim of the project was to produce a functioning system that can serve as a \textit{cryptographically secure} resource server for the \acrfull{dcs} and its users.
Production of the service required determining the end users of the service and the needs of the \acrshort{dcs}. To achieve this, the project aimed to identify all the users in the scenario of the \acrshort{dcs} with an analysis of the structure of staff and students (see Appendix \ref{appendix:roles_users}).
\vskip 0.5em
To provide a secure resource server, resources would have to be encrypted at-rest. However, \textit{symmetric} encryption introduces difficulties of securely distributing pre-shared keys. Meaning that a truly secure implementation cannot simply use an Access Control system on top of an \acrlong{aes} \citep{Daemen2003} service.

The service would instead implement an \acrfull{abe} \citep{Waters2011} system and needed to provide an enrolment process of bringing new users into the system and issuing new cryptographic keys. This would include determining the validity of a user's attributes and identifying an authority that can securely be tasked with performing said validation.
\vskip 0.5em
It was determined that creating an \acrshort{abe} library was beyond the project scope, so the project would have to integrate an external library into the system. Determination of which \acrshort{abe} library would focus on the extensibility of the library as well as evidence of the library in use for similar scenarios as a departmental resource server.
\vskip 0.5em
The project would also design Case Studies for the system to determine if all the needs of the \acrshort{dcs} were met by the implementation. Additionally, a risk assessment of the product would also be carried out to identify potential risks in the system and to verify the security of the final implementation.
