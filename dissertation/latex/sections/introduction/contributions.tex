\section{Contributions}

\subsection{Policy Language}

The project produced a formal language definition of a policy language for the ABE system. This language allows the ABE system to enforce strict static typing across all policies, ensuring reliable and consistent use over all resources.\\
The formal definition includes the syntax and types for the language along with the typing rules that are enforced on a policy before encryption of a resource can be processed. The definition also includes substitution rules and the required big step semantics for the language, with a final interpretation definition to intepret the policy language to the Python bindings used in the product.

\subsection{Software}

Production of the resource service required the defining of two servers (as described above) and the creation of the services running on each. Additionally, a third product was created in order to ease use of the service for users in the form of a local web GUI client that handles communication with the resource service for the user.
\vskip 0.5em
The first server was defined as an offline master key server and would be tasked with initiating the ABE system and provisioning the master private key for the entire service. This server would remain offline from the point of complete installation, ensuring that the key is provisioned after the server enters the offline state and protecting the master private key from external internet threats.\\
The second server was defined as an online, 'dumb' storage service for the encrypted resources, only ever storing the ciphertext binary blobs with no way to decrypt or read the uploaded resources. This storage service would also be responsible for distributing the master public key, which would be manually uploaded from the master key server using some form of offline method.\\
The third product was designed to be ran locally on a user's device and then connect to the storage service in order to provide the services on offer through a simple GUI. The user provides this local client with their private user key and the service uses it to \textit{locally} decrypt resources, ensuring the key never leaves the user's device. For encrypting, the local client retrieves the master public key from the resource server and performs encryption on any required resources using the public key and the relevant policies, as defined by the user.
