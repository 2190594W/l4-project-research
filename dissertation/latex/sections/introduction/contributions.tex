\section{Contributions}
\label{sec:intro_contrib}

\subsection{Policy Language}
\label{subsec:intro_pol_lang}

The project produced a formal language definition of a policy language for the \acrshort{abe} system, known as \thePolicyLang. This language, allows the \acrshort{abe} system to clearly define the structure of all policies, ensuring reliable and consistent use over all resources.\\
The \thePolicyLang definition includes the syntax and types, along with the typing rules that are enforced on a policy before encryption of a resource can be processed. The definition also includes substitution rules and the required big step semantics for the language, with a final interpretation definition to interpret \thePolicyLang to the Python bindings used in the product.

\subsection{Software}
\label{subsec:intro_software}

Production of the resource service required the defining of two servers (as described above) and the creation of the services running on each. Additionally, a third product was created to ease use of the service for users, in the form of a local web GUI client that handles communication with the resource service for the user.
\vskip 0.5em
An offline \acrfull{mks} would be tasked with initiating the \acrshort{abe} system and provisioning the master private key for the entire service. This server would remain offline from the point of complete installation, ensuring that the key is provisioned after the server enters the offline state and protecting the master private key from external threats.\\
A separate online, 'dumb' storage service for all encrypted resources, would only ever store the ciphertext binary blobs with no method of decrypting or reading any uploaded resources. This storage service would also be responsible for distributing the master public key, which could be manually uploaded from the \acrshort{mks} using a physical, offline transfer.\\
An end user product can be ran locally on a user's device and then connected to the storage service to provide the services on offer through a simple GUI\@. The user could then provide this local client with their private user key and the service would use it to \textit{locally} decrypt any resources, ensuring the key never leaves the user's device. For encrypting, the local client retrieves the master public key from the \acrfull{prs} and performs encryption on any required resources using the public key and the relevant policies, as defined by the user.
