\section{Access Control}
\label{sec:bkgr_acc_ctrl}

\subsection{Overview}
\label{subsec:acc_ctrl_over}

Access Control encompasses the authentication and authorization of users in a system, as well as the audit process for said system. In the scope of this project, we only consider the authorization part of Access Control, since the service does not authenticate users before permitting the upload or download of resources, because the encryption of resources protects against rogue access of data. The \nameref{ch:design} chapter discusses this decision further, but since users are not authenticated, it is also difficult to audit the service due to lack of information on users.
\vskip 0.5em
Regardless, Access Control is vital to maintaining the security of online services as it allows users to be granted access to certain routes, documents and data, but denied access to others. This most often takes the form of Role Base Access Control (RBAC), where users are assigned a role such as `\textit{Customer}', `\textit{Staff}' or `\textit{Admin}' with each role having a different set of permissions.\\
For example, a user with the `\textit{Customer}' role would not have access to the organisation's internal documents, whereas a user with the `\textit{Staff}' role would. Similarly, a user with the `\textit{Staff}' role would not necessarily have access to a service's user details, whereas a user with the `\textit{Admin}' role would.

\subsection{Models}
\label{subsec:acc_ctrl_models}

\newcommand\todonote[1]{\textcolor{red}{#1}}
Advanced RBAC systems can even grant a user multiple roles with different permissions, although a more common practice is to have each role grant a subset of the permissions granted by another role. For example, a  `\textit{Superadmin}' role would have all permissions, with a lower `\textit{Admin}' role having a subset of those permissions; such as not having permission to edit or delete other `\textit{Admin}' accounts.\\
Other Access Contol models do exist and generally all models fall into one of two categories, \textbf{capability-based} models or \textbf{access control lists-based} (ACL-based) models; where RBAC represents a capability-based model and ABE represents an ACL-based model.
\vskip 0.5em
\textit{Capability-based models} are based on the ability of a user to prove possession of an unforgeable (\todonote{\textit{Reference Wikipedia?}}) reference or \textit{capability} that aligns with the references of the system they are authorizing against. In the case of RBAC, the user proves that they have the \textit{role} required for access via some immutable piece of data such as a cookie or JSON Web Token.\\
\textit{ACL-based models} are instead based around a user's identity appearing in a list assigned to or embedded within the object, data or route they are attempting to access. In the case of ABE, the policy embedded in an encrypted resource serves the role of the list and a user's private key contains their identity which must then be parsed by the policy.
