\section{Public Key Infrastructure}
\label{sec:bkgr_pub_key_infr}

Public key cryptography is briefly discussed in the \Cref{sec:bkgr_encryption}, and is originally attributed to \citet{Diffie1976} where a full description of the pubic key encryption \& decryption processes can be found.

When employed in a real-world product, public key cryptography requires an infrastructure for the distribution of public keys. This is an absolute requirement, as an end user must be able to trust that the public key they have received is genuinely the public key of the system or person they are communicating with.
\vskip 0.5em
In the case of websites, each server hosting a website must communicate with a browser via the \acrshort{tls} protocol \citep{Rescorla2018}, verifying the communication with a \acrshort{tls}/\acrshort{ssl} certificate. The browser then validates that certificates it receives belong to the website they are trying to visit, by verifying with a \acrfull{ca}.

In the case of the Internet, \acrshort{ca}s are independent, pre-approved companies that securely authorise that a server is genuinely owned and operated by/for a website, with tools such as DNS validation \citep{Hunt2001}.
