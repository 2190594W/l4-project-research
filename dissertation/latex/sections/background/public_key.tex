\section{Public Key Infrastructure}
\label{sec:bkgr_pub_key_infr}

Public key cryptography is briefly discussed in the \nameref{sec:bkgr_encryption} section above and is originally attributed to \citet{Diffie1976} where a full description of the pubic key encryption \& decryption processes can be found.\\
When employed in a real-world product, public key cryptography requires an infrastructure for the distribution of public keys. This is an absolute requirement, as an end user must be able to trust that the public key they have recieved is genuinely the public key of the system or person they are communicating with.\\
Chatrooms or messaging services have central servers that process this distribution of public keys, in order to ensure any party of the communication may verify public keys against a single authority. If a public key fails verification with that authority, then a user may assume that the public key they have received is a rogue or otherwise invalid key.
\vskip 0.5em
In the case of websites and the TLS protocol \citep{Rescorla2018}, each server hosting a website must communicate with a browser via a the TLS protocol, verifying the communication with a TLS/SSL certificate. The browser requires the ability to then validate that a certificate it has received, does actually belong to the website they are trying to visit. Hence, Certificate Authorities (CAs) serve this purpose for the Internet, by validating that a served certificate is genuinely from the website it claims to be.\\
In the case of the Internet, CAs are independent, pre-appoved companies that securely authorise that a server is genuinely owned and operated by/for a website with tools such as DNS validation \citep{Hunt2001}. These CAs are thus the full and final authorisation that a browser is genuinely communicating securely with the website the user intended to visit.
\vskip 0.5em
A deployed resource server system must also be able to facilitate the verification of its distributed public key or risk users encrypting resources with a rogue key. Whilst such a deployment would employ TLS for communication with devices \textemdash\ verifying its public TLS/SSL certificate with a public CA \textemdash\ the public key itself cannot directly be verified by the same CA.\\
Thus, a deployment of the \theResServer system must instead employ the public resource server as the CA for the public key on behalf of the master key server \textemdash\ which would not be directly accessible to users. This is still secure and verifiable, as when a user communicates with the \theResServer system, they do so via TLS and have verified via a CA that they are genuinely communicating with the \theResServer system. They can then download the public key from the public resource server (still over TLS) and separately validate their copy of the public key against a checksum hosted on the public resource server.
