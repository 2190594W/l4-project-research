\section{Resource Servers}
\label{sec:bkgr_res_srvr}

Resource servers are a central operating component of any organisation and are often considered `mission critical' due to the reliance users have on accessing shared resources. More and more organisations are implementing cloud storage into their day-to-day operations through products such as OneDrive, Google Drive or Dropbox \citep{Thorpe2018}.\\
These services allow for fast and simple sharing of resources between colleagues and even external parties with an added benefit of backup facilities for user resources. Users can also collaborate on resources that have been uploaded and choose to share resources with individuals or whole teams. The full backup features of such services should be considered separately though, as they do not fall under the feature-umbrella of resource servers but rather backup services.
\vskip 0.5em
Given the organisational uses for resource servers or cloud storage, a resource server product should allow users in an organisation to upload and download their resources \textemdash\ storing a copy of their resource on the server. Users should also be able to share their uploaded resources with other users in their organisation either by roles \& teams or more granularly on a per-user basis.\\
Resources must also be kept secure both in-transit and at-rest. For transmission of resources to a resource server, this is solved by \acrshort{tls} communication as described in \nameref{sec:bkgr_pub_key_infr} above. However, at-rest protection is more complex as it usually requires encryption of resources with a specification such as \acrshort{aes} \citep{Daemen2003}, an implementation of \textit{symmetric} encryption that relies on a single key for encryption \& decryption.
\vskip 0.5em
Cloud storage providers do not offer the same level of security as that of a dedicated, \acrshort{abe}-encrypted resource server due to the \textit{symmetric} nature of \acrshort{aes}\@. OneDrive, Google Drive and Dropbox all offer at least 128-bit \acrshort{aes} encryption at-rest for \textbf{business} accounts \citet{Winder2018} yet each provider must store the key(s) used for encryption themselves.\\
This represents a security risk as each provider is responsible for ensuring that an external attacker is not able to acquire the encryption keys and that access control to said keys, is sufficiently restricted. By comparison, an \acrshort{abe}-encrypted resource server does not store the keys used for decryption and is implicitly secured against resource information breaches \textemdash\ since the server itself does not have access to any resource's information.
