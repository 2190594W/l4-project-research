\section{Resource Servers}
\label{sec:bkgr_res_srvr}

Resource servers are a central operating component of any organisation and organisations continue to implement cloud storage for this purpose, into their day-to-day operations through products such as OneDrive, Google Drive or Dropbox \citep{Thorpe2018}.

These services allow for fast and simple sharing of resources between colleagues and even external parties with an added benefit of backup facilities for user resources. We do not consider backup features for this project, as they do not fall under the `feature-umbrella' of resource servers but rather backup services.
\vskip 0.5em
Given the organisational requirements, a resource server product should allow users in an organisation to upload and download their resources securely. Users should also be able to share their uploaded resources with other users in their organisation either by roles \& teams or more granularly on a per-user basis.

Resources must be kept secure both in-transit and at-rest. For transmission of resources to a resource server, this is solved by \acrshort{tls} communication as described in \Cref{sec:bkgr_pub_key_infr}. However, at-rest protection is more complex as it requires encryption of resources with a specification such as the \acrlong{aes} \citep{Daemen2003}.
\vskip 0.5em
OneDrive, Google Drive and Dropbox all offer at least 128-bit \acrshort{aes} encryption at-rest for \textbf{business} accounts \citet{Winder2018} yet each provider must store the key used for encryption themselves.

This represents a security risk, as each provider has control over both ciphertext and decryption key. By comparison, an \acrshort{abe} resource server does not store the keys used for decryption and is thus, implicitly protected against resource information breaches.
