\section{Interpretation}\label{sec:interpretation}

\begin{figure}[ht]
  \centering
\begin{align*}
  \textsc{PolicyLang}&\rightarrow\textsc{Python (PyOpenABE)}\\
  \interpB{\TyInt}&\Coloneqq\text{\ttfamily int}\\
  \interpB{\mathbb{B}}&\Coloneqq\text{\ttfamily bool}\\
  \interpB{\mathbb{D}}&\Coloneqq\text{\ttfamily datetime.date}\\
  \interpB{\mathbb{S}}&\Coloneqq\text{\ttfamily str}\\
  \interpB{\mathbb{L}}&\Coloneqq\text{\ttfamily list}\\
  \interpB{n}&\Coloneqq\text{\ttfamily int(}n\text{\ttfamily)}\\
  \interpB{\EnumFalse}&\Coloneqq\text{\ttfamily False}\\
  \interpB{\EnumTrue} &\Coloneqq\text{\ttfamily True}\\
  \interpB{d}&\Coloneqq\text{\ttfamily datetime.date(}d\text{\ttfamily)}\\
  \interpB{s}&\Coloneqq\text{\ttfamily str(}s\text{\ttfamily)}\\
  \interpB{l}&\Coloneqq\text{\ttfamily [}l\text{\ttfamily]}\\
  \interpB{\exprOr{a}{b}}  &\Coloneqq\interpB{a}\,\text{\ttfamily or}\,\interpB{b}\\
  \interpB{\exprAnd{a}{b}} &\Coloneqq\interpB{a}\,\text{\ttfamily and}\,\interpB{b}\\
  \interpB{\exprGT{a}{b}}  &\Coloneqq\interpB{a}\,\text{\ttfamily \textgreater}\,\interpB{b}\\
  \interpB{\exprLT{a}{b}}  &\Coloneqq\interpB{a}\,\text{\ttfamily \textless}\,\interpB{b}\\
  \interpB{\exprEQ{a}{b}}  &\Coloneqq\interpB{a}\,\text{\ttfamily ==}\,\interpB{b}\\
  \interpB{\exprGTE{a}{b}}  &\Coloneqq\interpB{a}\,\text{\ttfamily \textgreater =}\,\interpB{b}\\
  \interpB{\exprLTE{a}{b}}  &\Coloneqq\interpB{a}\,\text{\ttfamily \textless =}\,\interpB{b}
\end{align*}
  \caption{\label{fig:interp}Interpretation Rules}
\end{figure}

In this final section we describe how we can interpret our language to another form, in this case concrete Java expressions.
Like substitution, interpretation recursively operates over each language statement and expression.
At each step if replaces the expression from our language with it's equivalent Java form.
If we interpret a language we do not need to provide operational semantics, the target language provides this.
